\documentclass[a4paper, 12pt]{article}

%%% SST LAB PROTOCOLL PREAMBLE
%%% 2019
%%%%%%%%%%%%%%%%%%%%%%%%%%%%%%%


%%% PACKAGES
%%%%%%%%%%%%%%%%%%%%%%%%%%%

\usepackage{amsmath}
\usepackage{pgfplots}
\usepackage{tikz}
\usepackage{tcolorbox}
\usepackage{graphicx}
\graphicspath{ {./graphics/} }
\usepackage{pdfpages}


%%% DOCUMENT GEOMETRY
%%%%%%%%%%%%%%%%%%%%%%%%%%%

\usepackage{geometry}
\geometry{
 a4paper,
 total={0.6180339887498948\paperwidth,0.6180339887498948\paperheight},
 top = 0.1458980337503154\paperheight,
 bottom = 0.1458980337503154\paperheight
 }
\setlength{\jot}{0.013155617496424828\paperheight}
\linespread{1.1458980337503154}

\setlength{\parskip}{0.013155617496424828\paperheight} % paragraph spacing


%%% COLORS
%%%%%%%%%%%%%%%%%%%%%%%%%%%

\definecolor{red1}{HTML}{f38181}
\definecolor{yellow1}{HTML}{fce38a}
\definecolor{green1}{HTML}{95e1d3}
\definecolor{blue1}{HTML}{66bfbf}
\definecolor{hsblue}{HTML}{0cb5df}
\definecolor{hsblueshade}{HTML}{b6e9f5}


%%% COMMANDS
%%%%%%%%%%%%%%%%%%%%%%%%%%%

\newcommand{\holine}{
  \noindent\rule{\textwidth}{0.618pt}\\[0.021286\paperheight]
}

\newcommand{\minisec}[1]{ \underline{\textit {#1} } \\[0.021286\paperheight]}

\newcommand{\plotfun}[3]{
  \vspace{0.021286\paperheight}
  \begin{center}
    \begin{tikzpicture}
      \begin{axis}[
        axis x line=center,
        axis y line=center,
        ]
        \addplot[draw=red1][domain=#2:#3]{#1};
      \end{axis}
    \end{tikzpicture}
  \end{center}
}

\tcbset{colback=white,colframe=red1!100!black,title=Note!,width=0.618\paperwidth,arc=0pt}
\newcommand{\notebox}[1]{

 \begin{center}
  \begin{tcolorbox}[]
   #1 
  \end{tcolorbox}
 
 \end{center} 
 
}

% END OF PREAMBLE

%%%%%%%%%%%%%%%%%%%%%%%%%%%%%%%%%%%%%

\begin{document}

%%%%%%%%%%%%%%%%%%%%%%%%%%%%%%%%%%%%%
  \includepdf{./titlepage/titlepage.pdf}
  \clearpage
  \setcounter{page}{1}
%%%%%%%%%%%%%%%%%%%%%%%%%%%%%%%%%%%%%

\section{Vorbereitungsaufgaben}

\subsection{}
\begin{figure}[H]
	\includegraphics[width=\textwidth]{1_1/TP_ideal}
\end{figure}

\begin{figure}[H]
	\includegraphics[width=\textwidth]{1_1/HP_ideal}
\end{figure}

\begin{figure}[H]
	\includegraphics[width=\textwidth]{1_1/BP_ideal}
\end{figure}

\subsection{}

\begin{figure}[H]
	\includegraphics[width=\textwidth]{1_2/TP}
\end{figure}

\begin{figure}[H]
	\includegraphics[width=\textwidth]{1_2/Ausgangsspektrum}
\end{figure}


Aus der grafischen Betrachtung der Spektren von Ein- und Ausgangssignal und
unter Berücksichtung der Übergangsfunktion des Tiefpasses lässt sich erkenn,
dass die Grenzfrequenz des Tiefpasses zwischen $\omega_0$ und $2 \,\omega_0$
liegen muss, um den Frequenzanteil bei $\omega_0$ durchzulassen und den bei
$2\,\omega_0$ auszulöschen. Der Faktor $K_1$ ist durch die Skalierung der
Amplitude beim Durchlauf des Tiefpasses gleich $1/2$.

Der Parameter $\phi$ beschreibt die Phasenverschiebung vom Ausgangssignal zum
Eingangssignal, welche sich beim idealen Tiefpass linear mit der Frequenz
ändert. Im Zeitbereich erkennt man dadurch eine Zeitverschiebung $t_0$, wodurch
$\phi = \omega_0 \cdot t_0$.

\subsection{}
Das System kann lineare Verzerrungen hervorrufen, da es Frequenzanteile
unterdrücken oder durchlassen kann pipipipipi

\subsection{}
Zufällige Signale können durch ihre \emph{Autokorrelation}, welche die Änhlichkeit eines
Signals mit sich selbst ausdrückt, charakterisiert werden. Idealerweise ist
diese Ähnlichkeit bei zufälligen Signalen bei jeder Verschiebung ungleich 0 nicht gegeben.

Ideales gaußverteiltes Rauschen besitzt zudem ein \emph{konstantes
Leistungsdichtespektrum}.

\subsection{}

Idealerweise kann die Übertragungsfunktion eines Systems durch Anregung des
Systems mit einem \emph{Dirac-Impuls} im Zeitbereich bestimmt werden. Der Dirac-Impuls
besitzt im Frequenzbereich eine konstante Amplitude (1) über alle Frequenzen.
Die Auswertung im Frequenzbereich ergibt daher (mit $\Delta(j\omega) = \mathcal{F} (\delta(t))$):
$$ U_2(j \omega) = G(j \omega) \cdot U_1(j \omega) = G(j \omega) \cdot \Delta(j
\omega) = G(j \omega)$$

Der demnach ermittelte Betragsgang am Ausgang ist somit gleich der
Übertragungsfunktion des Systems. Praktisch ist diese Methode jedoch kaum
möglich, da es schwierig ist einen zeitlich unendlich kurzen Impuls von
unendlicher Amplitude (hohe Energie) zu
erzeugen und gleichzeitig Nichtlinearitäten,
Trägheit/Verzögerungen und Belastbarkeit des zu messenden Systems zu berücksichtigen.

Als praktischere Alternative zum Dirac-Impuls kann man einen \emph{Sprung} auf das
System geben und 

Eine weitere Möglichkeit ist die diskrete Analyse des Systems bei bestimmten
Frequenzen, z.B. durch Anregung mit einem \emph{Wobble-Sinus} und die Aufnahme von
Messpunkten in geeigneten Abständen. Nachteil ist hier die begrenzte Auflösung,
die unter Umständen problematisch sein kann.

\section{Versuchsaufgaben}


\end{document}
