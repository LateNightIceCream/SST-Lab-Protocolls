\documentclass[a4paper, 12pt]{article}

%%% SST LAB PROTOCOLL PREAMBLE
%%% 2019
%%%%%%%%%%%%%%%%%%%%%%%%%%%%%%%


%%% PACKAGES
%%%%%%%%%%%%%%%%%%%%%%%%%%%

\usepackage{amsmath}
\usepackage{pgfplots}
\usepackage{tikz}
\usepackage{tcolorbox}
\usepackage{graphicx}
\graphicspath{ {./graphics/} }
\usepackage{pdfpages}


%%% DOCUMENT GEOMETRY
%%%%%%%%%%%%%%%%%%%%%%%%%%%

\usepackage{geometry}
\geometry{
 a4paper,
 total={0.6180339887498948\paperwidth,0.6180339887498948\paperheight},
 top = 0.1458980337503154\paperheight,
 bottom = 0.1458980337503154\paperheight
 }
\setlength{\jot}{0.013155617496424828\paperheight}
\linespread{1.1458980337503154}

\setlength{\parskip}{0.013155617496424828\paperheight} % paragraph spacing


%%% COLORS
%%%%%%%%%%%%%%%%%%%%%%%%%%%

\definecolor{red1}{HTML}{f38181}
\definecolor{yellow1}{HTML}{fce38a}
\definecolor{green1}{HTML}{95e1d3}
\definecolor{blue1}{HTML}{66bfbf}
\definecolor{hsblue}{HTML}{0cb5df}
\definecolor{hsblueshade}{HTML}{b6e9f5}


%%% COMMANDS
%%%%%%%%%%%%%%%%%%%%%%%%%%%

\newcommand{\holine}{
  \noindent\rule{\textwidth}{0.618pt}\\[0.021286\paperheight]
}

\newcommand{\minisec}[1]{ \underline{\textit {#1} } \\[0.021286\paperheight]}

\newcommand{\plotfun}[3]{
  \vspace{0.021286\paperheight}
  \begin{center}
    \begin{tikzpicture}
      \begin{axis}[
        axis x line=center,
        axis y line=center,
        ]
        \addplot[draw=red1][domain=#2:#3]{#1};
      \end{axis}
    \end{tikzpicture}
  \end{center}
}

\tcbset{colback=white,colframe=red1!100!black,title=Note!,width=0.618\paperwidth,arc=0pt}
\newcommand{\notebox}[1]{

 \begin{center}
  \begin{tcolorbox}[]
   #1 
  \end{tcolorbox}
 
 \end{center} 
 
}

% END OF PREAMBLE

%%%%%%%%%%%%%%%%%%%%%%%%%%%%%%%%%%%%%

\begin{document}

%%%%%%%%%%%%%%%%%%%%%%%%%%%%%%%%%%%%%
  \includepdf{./titlepage/titlepage.pdf}
  \clearpage
  \setcounter{page}{1}
%%%%%%%%%%%%%%%%%%%%%%%%%%%%%%%%%%%%%

\section{Vorbereitungsaufgaben}

\subsection{}
Die Grundfrequenz der dargestellten (periodischen) Datenfolge ist

$$f = \frac{1}{T} = \frac{1}{1 \, \si{\micro\second}} = 1 \, \si{\mega\hertz}$$

\subsection{}

\begin{figure}[H]
\begin{center}
  \begin{circuitikz}
    \draw (0,0) to[R, l=$R$, o-] (2,0) to[L, l=$L$] (4,0) -- (7,0) to[short, -o]
    (9,0);
    \draw (5,0) to[R, l=$G$, *-*] (5,-3);
    \draw (7,0) to[C, l=$C$, *-*] (7,-3);
    \draw (0,-3) to[short, o-o] (9,-3);
  \end{circuitikz}
\end{center}

\caption{Ersatzschaltbild einer Kupferdoppelader}
\end{figure}

Die Kupferdoppelader kann im Ersatzschaltbild durch einen Widerstand $R$, eine
Induktivität $L$, einem Leitwert $G$ und eine Kapazität $C$
modelliert werden, welche alle proportional zur Länge der Leitung sind. Die auf
die Leitungslänge bezogenen
Parameter des Ersatzschaltbildes werden in der Leitungstheorie auch
\emph{Leitungsbeläge} genannt. \\ Der Widerstand $R$ der Leitung entsteht durch den spezifischen
Widerstand des Materials $\rho$, der Querschnittsfläche $A$ und der Länge $l$ der Leitung
nach $$R = \rho \cdot \frac{l}{A}, $$ wobei zu beachten ist, dass 
die gesamte Leitungslänge, also beide Adern bei der Doppelader, als Länge in die
Berechnung eingeht. Zudem ist der Widerstandsbelag frequenzabhägig (Skineffekt).

Da die gesamte Doppelader eine Spule mit nur einer Windung darstellt, tritt eine
Induktivität $L$ auf. 
$$ L = (N^2) \cdot \mu \cdot \frac{A}{l} $$
Praktisch wird weiterhin zwischen einer äußeren Induktivität (magnet.
Eigenschaften, Geometrie des
Leiters) und einer inneren Induktivität (Wechselfelder innerhalb des Leiters, Skineffekt) unterschieden

Die Kapazität $C$ entsteht durch die Kondensatorwirkung der zwei dicht
beeinanderliegenden Adern.
$$ C = \epsilon \cdot \frac{A}{d}$$

Die dielektrischen Verluste zwischen beiden Adern werden durch den Leitwert $G$
(Ableitungsbelag) gekennzeichnet, welcher auch über den Verlustwinkel $\delta$
der Kapazität des Dielektrikums bestimmt ist.

Durch frequenzabhägige Elemente ($L$, $C$) im Ersatzschaltbild wird die
Übertragungsfunktion der Doppelader ebenfalls frequenzabhägig und weicht damit
von einer idealen, frequenzunabhägigen Leitung ab. Aus dem
Ersatzschaltbild lässt sich das Tiefpassverhalten der Doppelader herleiten.
Außerdem können durch Bildung eines Schwingkreises unerwünschte Resonanzerscheinungen auftreten.

\subsection{}

\begin{figure}[H]
	\includegraphics[width=\textwidth]{1_3/Si}
  \caption{Teil des Spektrums der periodischen 1010-Folge mit der Si-Funktion als Einhüllende}
\end{figure}

\begin{figure}[H]
	\includegraphics[width=\textwidth]{1_3/Si_filtered}
  \caption{Beispielhaftes Spektrum der periodischen 1010-Folge mit TP-Filtercharakteristik der
    Doppelader}
\end{figure}


Da hier eine periodische Folge gewählt wurde, entsteht ein diskretes
Amplitudenspektrum, welches durch Fouriersynthese wieder in die Ursprungsfolge
umgewandelt werden kann. Fehlen nun einige Spektrallinien  durch die
Tiefpasscharakteristik der Doppelader (in Abb. 3 grau gekennzeichnet), weicht die Folge nach
der Synthese von der Originalfolge ab (Abb. 4).

\begin{figure}[H]
	\includegraphics[width=\textwidth]{1_3/loss}
  \caption{ Auswirkung des spektralen Verlusts im Zeitbereich }
\end{figure}


\subsection{}

\subsection{}

\subsection{}

\section{Versuchsaufgaben}

\subsection{}

\subsection{}

\end{document}
