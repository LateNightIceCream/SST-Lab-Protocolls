\documentclass[a4paper, 12pt]{article}

%%% SST LAB PROTOCOLL PREAMBLE
%%% 2019
%%%%%%%%%%%%%%%%%%%%%%%%%%%%%%%


%%% PACKAGES
%%%%%%%%%%%%%%%%%%%%%%%%%%%

\usepackage{amsmath}
\usepackage{pgfplots}
\usepackage{tikz}
\usepackage{tcolorbox}
\usepackage{graphicx}
\graphicspath{ {./graphics/} }
\usepackage{pdfpages}


%%% DOCUMENT GEOMETRY
%%%%%%%%%%%%%%%%%%%%%%%%%%%

\usepackage{geometry}
\geometry{
 a4paper,
 total={0.6180339887498948\paperwidth,0.6180339887498948\paperheight},
 top = 0.1458980337503154\paperheight,
 bottom = 0.1458980337503154\paperheight
 }
\setlength{\jot}{0.013155617496424828\paperheight}
\linespread{1.1458980337503154}

\setlength{\parskip}{0.013155617496424828\paperheight} % paragraph spacing


%%% COLORS
%%%%%%%%%%%%%%%%%%%%%%%%%%%

\definecolor{red1}{HTML}{f38181}
\definecolor{yellow1}{HTML}{fce38a}
\definecolor{green1}{HTML}{95e1d3}
\definecolor{blue1}{HTML}{66bfbf}
\definecolor{hsblue}{HTML}{0cb5df}
\definecolor{hsblueshade}{HTML}{b6e9f5}


%%% COMMANDS
%%%%%%%%%%%%%%%%%%%%%%%%%%%

\newcommand{\holine}{
  \noindent\rule{\textwidth}{0.618pt}\\[0.021286\paperheight]
}

\newcommand{\minisec}[1]{ \underline{\textit {#1} } \\[0.021286\paperheight]}

\newcommand{\plotfun}[3]{
  \vspace{0.021286\paperheight}
  \begin{center}
    \begin{tikzpicture}
      \begin{axis}[
        axis x line=center,
        axis y line=center,
        ]
        \addplot[draw=red1][domain=#2:#3]{#1};
      \end{axis}
    \end{tikzpicture}
  \end{center}
}

\tcbset{colback=white,colframe=red1!100!black,title=Note!,width=0.618\paperwidth,arc=0pt}
\newcommand{\notebox}[1]{

 \begin{center}
  \begin{tcolorbox}[]
   #1 
  \end{tcolorbox}
 
 \end{center} 
 
}

% END OF PREAMBLE

%%%%%%%%%%%%%%%%%%%%%%%%%%%%%%%%%%%%%

\begin{document}

%%%%%%%%%%%%%%%%%%%%%%%%%%%%%%%%%%%%%
  \includepdf{./titlepage/titlepage.pdf}
  \clearpage
  \setcounter{page}{1}
%%%%%%%%%%%%%%%%%%%%%%%%%%%%%%%%%%%%%

\section{Vorbereitungsaufgaben}


\subsection{}
\includegraphics[width=\textwidth]{./1_1/1_1_function}


\begin{gather}
	\intertext{Hier gilt}
	x(t) = x(-t), \label{eq:1}\\
	\intertext{weshalb x(t) eine gerade Funktion ist. Damit ist}
	\nonumber
	b_n = \frac{2}{T} \int_T{ x(t) \cdot \sin{( n \omega_0 t)}  \dif t} = 0\\[\smallvert]
	\nonumber
	a_n = \frac{2}{T} \int_T{ x(t) \cdot \cos{( n \omega_0 t)}  \dif t} \\
	\nonumber
	= \frac{2}{T} 
	\left[
		\int_{-T/2}^{-\tau/2}{...}	 +
		\int_{-\tau/2}^{\tau/2}{...}+
		\int_{\tau/2}^{T/2}{...}
	\right]\\
	\intertext{mithilfe von Gl. \ref{eq:1}:}
	\nonumber
	a_n = \frac{2}{T}
	\left[
		2 \int_{0}^{\tau/2}{A \cdot \cos{( n \omega_0 t)}\dif t}+
		2 \int_{\tau/2}^{T/2}{-A \cdot \cos{( n \omega_0 t)}\dif t}
	\right]\\
	\nonumber
	= \frac{4 A}{T}
	\left[
		\int_{0}^{\tau/2}{\cos{( n \omega_0 t)}  \dif t} -
		\int_{\tau/2}^{T/2}{\cos{( n \omega_0 t)}  \dif t}
	\right]\\
	\nonumber
	= \frac{4 A}{T} \cdot \frac{1}{n \omega_0}
	\left[
		\sin{( n \omega_0 t )} \bigg\rvert_{0}^{\tau/2} -
		\sin{( n \omega_0 t )} \bigg\rvert_{\tau/2}^{T/2}
	\right]\\
	\intertext{mit $\omega_0 = \frac{2 \pi}{T}$:}
	\nonumber
	a_n = 
	\frac{4 A \cdot T}{T \cdot n 2 \pi}
	\left[ 
		\sin{(n \frac{2 \pi}{T} \frac{\tau}{2})} - 
		\left(
			\sin{(n \frac{2 \pi}{T} \frac{T}{2})} -
			\sin{(n \frac{2 \pi}{T} \frac{\tau}{2})}
		\right)
	\right]\\
	\nonumber
	= \frac{2 A}{n \pi}
	\left[ 
		\sin{(n \pi \frac{\tau}{T})} - 
		\underbrace{\sin{(n \pi)}}_{=0} +
		\sin{(n \pi \frac{\tau}{T})}
	\right]
\end{gather}

\eqbox{a_n = \frac{4 A}{n \pi} \cdot \sin{(n \pi \frac{\tau}{T})}}{0.382\textwidth}

\holine{0.618\textwidth}
%\noindent \textit{Berechnung des Gleichanteils}
\begin{gather*}
	a_0 = \frac{1}{T} \int_T{x(t) \dif t}\\
	\intertext{mithilfe von Gl. \ref{eq:1}:}
	a_0 = \frac{2}{T/2} \left[ \int_{0}^{\tau/2}{...} + \int_{\tau/2}^{T/2}{...} \right]\\
	= \frac{4 A}{T} \left[ t \bigg\rvert_{0}^{\tau/2} - t \bigg\rvert_{\tau/2}^{T/2}\right]\\
	= \frac{4 A}{T} \left[ \frac{\tau}{2} - \left( \frac{T}{2} - \frac{\tau}{2} \right)  \right]\\
	= \frac{4 A}{T} \left[ \tau - \frac{T}{2} \right]\\
	= 4 A \left[ \frac{\tau}{T} - \frac{1}{2} \right]
\end{gather*}

\eqbox{\frac{a_0}{2} = 2 A \left[ \frac{\tau}{T} - \frac{1}{2} \right]}{0.382\textwidth}


\holine{\textwidth}

\begin{gather*}
	\intertext{Für das Tastverhältnis $\frac{\tau}{T} = 0.5$  gilt:}
	\frac{a_0}{2} = 2 A \left[ \frac{1}{2} - \frac{1}{2}\right] = 0,\\
	a_n = \frac{4 A}{n \pi} \cdot \sin{(n \pi \frac{1}{2})} = \frac{4 A}{n \pi} \cdot \sin{(n \frac{\pi}{2})}\\
	a_n = \frac{4 A}{n \pi} \cdot (-1)^{n+1}
\end{gather*}

\begin{figure}[H]
  \includegraphics[width=\textwidth]{1_1/1_1_AS}
	\caption{Betragsspektrum von x(t) für $\frac{\tau}{T} = 0.5$}
\end{figure}

%%%%%%%%%%%%%%%%%%%%%%%
\subsection{}

\begin{figure}[H]
  \includegraphics[width=\textwidth]{1_2/1_2_function}
	\caption{x(t) mit dem Gleichanteil X0 im Zeitbereich}
\end{figure}

\begin{gather*}
  \intertext{Auswirkungen im Frequenzbereich:}
  a_n = \frac{2}{T} \int_{T}{(x(t) + X0) \cdot \cos{(n \omega_0 t)}\dif t}\\
      = \frac{2}{T} \left[  \int_{T}{x(t) \cos{(n\omega_0 t)} \dif t}+
        \underbrace{\int_T{X0 \cdot
          \cos{(n \omega_0 t)} \dif t}}_{=0} \right]\\
  a_n = \frac{2}{T} \int_{T}{x(t) \cdot \cos{(n \omega_0 t)}\dif t}\\
  \intertext{$\rightarrow$ Im Frequenzbereich finden keine Änderungen statt.}
  \intertext{$\rightarrow$ Nur der Gleichanteil ändert sich}
\end{gather*}

\subsection{}

\begin{gather*}
	\intertext{Für das Tastverhältnis $\frac{\tau}{T} = 0.25$  gilt:}
	\frac{a_0}{2} = 2 A \left[ \frac{1}{4} - \frac{1}{2}\right] = - A ,\\
	a_n = \frac{4 A}{n \pi} \cdot \sin{(n \pi \frac{1}{4})}\\
	a_n = \frac{4 A}{n \pi} \cdot \sin{(n \frac{\pi}{4})}
\end{gather*}

\begin{figure}[H]
  \includegraphics[width=\textwidth]{1_3/1_3_Reihe}
	\caption{x(t) als Summe der Einzelschwingungen}
\end{figure}


\subsection{}
\begin{figure}[H]
  \includegraphics[width=\textwidth]{1_4/1_4_function}
\end{figure}



\begin{gather}
	\intertext{Es gilt:}
	x(-t) = -x(t),\\
	\intertext{weshalb $x(t)$ eine ungerade Funktion ist. Damit ist}
	\nonumber
	a_n = \frac{2}{T} \int_T{x(t) \cdot \cos{(n \omega_0 t)} \dif t} = 0\\
  \intertext{und}
	 \nonumber
  a_0 = \frac{1}{T} \int^{T/2}_{-T/2}{x(t) \dif t} = 0\\[\smallvert]
\nonumber
	b_n = \frac{2}{T} \int_T{x(t) \cdot \sin{(n \omega_0 t)} \dif t}\\
 	\nonumber
	= \frac{2}{T}
		\int_{-T/2}^{T/2}{ \frac{A}{T/2} \cdot t \cdot \sin{(n \omega_0 t)} \dif t}\\
	\nonumber
	= \frac{4 A }{T^2}
		\int_{-T/2}^{T/2}{ t \cdot \sin{(n \omega_0 t)} \dif t}\\
	\intertext{Partielle Integration:}
	\nonumber
	u = t, u' = 1\\
	\nonumber
	v' = \sin{(n\omega_0 t)}, v = - \frac{1}{n \omega_0} \cdot \cos{(n\omega_0 t)}\\
	\nonumber
	b_n = \frac{4 A}{T^2} \left[ - \frac{1}{n \omega_0} \cdot \left[ t \cos{n\omega_0 t} \right]\bigg\rvert_{-T/2}^{T/2} + \frac{1}{n \omega_0} \int_{-T/2}^{T/2}{ \cos(n \omega_0 t) \dif t  } \right]\\
	\intertext{mit $\omega_0 = \frac{2 \pi}{T}$}
	\nonumber
	= \frac{4 A}{T^2} \left[ 
	- \frac{1}{n \omega_0} \cdot \left[ \frac{T}{2} \cos{(n \frac{2 \pi T}{T 2})}-(-\frac{T}{2} \cos{(- n \frac{2 \pi T}{T 2})}) \right]+ \frac{1}{n^2 \omega_0^2}\left[ \sin{(n \omega_0 t)} \right]\bigg\rvert_{-T/2}^{T/2}
	\right]\\
	\nonumber
	= \frac{4 A}{T^2} \left[
	- \frac{1}{n \omega_0} \cdot \left[ \frac{T}{2} \cos{(n \pi)} + \frac{T}{2} \cos{(n \pi)} \right]+ \frac{1}{n^2 \omega_0^2}\underbrace{\left[ \sin{(n \frac{2 \pi}{T} t)} \right]\bigg\rvert_a{-T/2}^{T/2}}_{=0}
	\right]\\
	\nonumber
	= - \frac{4 A}{T^2 n \omega_0} \cdot \frac{2 T }{2} \left[
		\cos{(n \pi)}
	\right]\\
	\nonumber
	= -\frac{2 A}{n \pi} (-1)^n
\end{gather}

\eqbox{b_n = \frac{2A}{n \pi} \cdot (-1)^{n+1}}{0.382\textwidth}

\begin{figure}[H]
  \includegraphics[width=\textwidth]{1_4/1_4_AS}
  \caption{Betragsamplitudenspektrum von x(t)}
\end{figure}

\begin{figure}[H]
  \includegraphics[width=\textwidth]{1_4/1_4_Phase}
  \caption{Phasenspektrum von x(t)}
\end{figure}

\section{Versuchsaufgaben}

\subsection{}

Diagramm eingestellte/gemessene amplitudenwerte

\begin{figure}[H]
  \includegraphics[width=\textwidth]{2_1/scope_0_blue}
  \caption{Screenshot des Sinussignals auf dem Oszilloskop}
\end{figure}

% 2.2
\subsection{}

\begin{gather*}
	\intertext{Analytische Bestimmung des Amplitudenspektrums, Aus 1.1 ergibt sich:}
  D = \frac{\tau}{T} = \frac{1}{2}\\
  \frac{a_0}{2} = 0
\end{gather*}

\begin{gather*}
	a_n = \frac{4 \cdot 2 \si{\volt}}{n \pi}\cdot (-1)^{n+1}\\
\end{gather*}

\begin{table}[H]
\begin{center}
\begin{tabular}{@{}cccc@{}}
\toprule
$f/\si{\kilo\hertz}$ & $a_n / \si{\volt}$ & $b_n / \si{\volt}$ \\ \midrule
1                      &  2.457     & 0     \\
2                      &  0     & 0     \\
3                      &  -0.849     & 0     \\
4                      &  0     & 0     \\
5                      &  0.509     & 0     \\
6                      &  0     & 0     \\
7                      &  -0.364     & 0     \\
8                      &  0    & 0     \\
9                      &  0.283     & 0     \\
10                     &  0    & 0     \\ \bottomrule
\end{tabular}
\end{center}
\caption{Fourierkoeffizienten $D = 0.5$}
\end{table}

\begin{figure}[H]
  \includegraphics[width=\textwidth]{2_2/scope_1_blue}
  \caption{Screenshot des Rechtecksignals ($D = 0.5$) auf dem Oszilloskop}
\end{figure}

\begin{figure}[H]
  \includegraphics[width=\textwidth]{2_2/scope_2_white}
  \caption{Screenshot des Amplitudenspektrums auf dem Oszilloskop}
\end{figure}

\begin{table}[H]
  \begin{center}
\begin{tabular}{@{}cccc@{}}
\toprule
$n_1-n_2$ & $\Delta a / \si{\volt}$ & $\Delta a / \si{\deci\bel}$ & Abweichung \\ \midrule
1-3       &                         & -10                         &            \\
3-5       &                         & -3,75                       &            \\
5-7       &                         & -2.812                      &            \\
7-9       &                         & -2.5                        &            \\ \bottomrule
\end{tabular}
\caption{Messwerte der Aufgabe 6.2}
\end{center}
\end{table}

Die Bestimmung von $\Delta a_n$ erfolgte durch
$$\Delta a_n = \frac{1}{a_1} \cdot 10^{\dfrac{\Delta a_{n_{dB}}  }{20}}$$

\begin{figure}[H]
  \includegraphics[width=\textwidth]{2_2/2_2_Amplivergleich}
  \caption{Vergleich der theoretischen (grau) und gemessenen (blau) Werte }
\end{figure}


% 2.3
\subsection{}

\begin{gather*}
  \intertext{Aus 1.3 folgt}
  \frac{a_0}{2} = -A = -2 \,\ \si{\volt}\\
  a_n = \frac{4 \cdot 2 \si{\volt}}{ n \pi } \cdot \sin{(n \pi \frac{1}{4})}
\end{gather*}

\begin{table}[H]
\begin{center}
\begin{tabular}{@{}cccc@{}}
\toprule
$f/\si{\kilo\hertz}$ & $a_n / \si{\volt}$ & $b_n / \si{\volt}$ \\ \midrule
1                      &  1.801     & 0     \\
2                      &  1.273     & 0     \\
3                      &  0.600     & 0     \\
4                      &  0     & 0     \\
5                      &  -0.360     & 0     \\
6                      &  -0.424     & 0     \\
7                      &  -0.257     & 0     \\
8                      &  0    & 0     \\
9                      &  0.200     & 0     \\
10                     &  0.255    & 0     \\ \bottomrule
\end{tabular}
\end{center}
\caption{Fourierkoeffizienten $D = 0.25$}
\end{table}

\begin{figure}[H]
  \includegraphics[width=\textwidth]{2_3/scope_3_blue}
  \caption{Screenshot des Rechtecksignals ($D = 0.25$) auf dem Oszilloskop}
\end{figure}

\begin{figure}[H]
  \includegraphics[width=\textwidth]{2_3/scope_4_white}
  \caption{Screenshot des Amplitudenspektrums auf dem Oszilloskop}
\end{figure}


% 2.4
\subsection{}

\begin{table}[H]
\begin{center}
\begin{tabular}{@{}cccc@{}}
\toprule
$f/\si{\kilo\hertz}$ & $a_n / \si{\volt}$ & $b_n / \si{\volt}$ \\ \midrule
1                      &  0       & 1.273     \\
2                      &  0     & -0.637     \\
3                      &  0     & 0.424     \\
4                      &  0     & -0.318     \\
5                      &  0     & 0.255     \\
6                      &  0     & -0.212     \\
7                      &  0     & 0.182     \\
8                      &  0    & -0.159     \\
9                      &  0     & 0.141     \\
10                     &  0    & -0.127     \\ \bottomrule
\end{tabular}
\end{center}
\caption{Fourierkoeffizienten $D = 0.25$}
\end{table}

\begin{figure}[H]
  \includegraphics[width=\textwidth]{2_4/scope_5_blue}
  \caption{Screenshot des Sägezahnsignals auf dem Oszilloskop}
\end{figure}

\begin{figure}[H]
  \includegraphics[width=\textwidth]{2_4/scope_6_white}
  \caption{Screenshot des Amplitudenspektrums auf dem Oszilloskop}
\end{figure}

\end{document}
