\documentclass[a4paper, 12pt]{article}

%%% SST LAB PROTOCOLL PREAMBLE
%%% 2019
%%%%%%%%%%%%%%%%%%%%%%%%%%%%%%%


%%% PACKAGES
%%%%%%%%%%%%%%%%%%%%%%%%%%%

\usepackage{amsmath}
\usepackage{pgfplots}
\usepackage{tikz}
\usepackage{tcolorbox}
\usepackage{graphicx}
\graphicspath{ {./graphics/} }
\usepackage{pdfpages}


%%% DOCUMENT GEOMETRY
%%%%%%%%%%%%%%%%%%%%%%%%%%%

\usepackage{geometry}
\geometry{
 a4paper,
 total={0.6180339887498948\paperwidth,0.6180339887498948\paperheight},
 top = 0.1458980337503154\paperheight,
 bottom = 0.1458980337503154\paperheight
 }
\setlength{\jot}{0.013155617496424828\paperheight}
\linespread{1.1458980337503154}

\setlength{\parskip}{0.013155617496424828\paperheight} % paragraph spacing


%%% COLORS
%%%%%%%%%%%%%%%%%%%%%%%%%%%

\definecolor{red1}{HTML}{f38181}
\definecolor{yellow1}{HTML}{fce38a}
\definecolor{green1}{HTML}{95e1d3}
\definecolor{blue1}{HTML}{66bfbf}
\definecolor{hsblue}{HTML}{0cb5df}
\definecolor{hsblueshade}{HTML}{b6e9f5}


%%% COMMANDS
%%%%%%%%%%%%%%%%%%%%%%%%%%%

\newcommand{\holine}{
  \noindent\rule{\textwidth}{0.618pt}\\[0.021286\paperheight]
}

\newcommand{\minisec}[1]{ \underline{\textit {#1} } \\[0.021286\paperheight]}

\newcommand{\plotfun}[3]{
  \vspace{0.021286\paperheight}
  \begin{center}
    \begin{tikzpicture}
      \begin{axis}[
        axis x line=center,
        axis y line=center,
        ]
        \addplot[draw=red1][domain=#2:#3]{#1};
      \end{axis}
    \end{tikzpicture}
  \end{center}
}

\tcbset{colback=white,colframe=red1!100!black,title=Note!,width=0.618\paperwidth,arc=0pt}
\newcommand{\notebox}[1]{

 \begin{center}
  \begin{tcolorbox}[]
   #1 
  \end{tcolorbox}
 
 \end{center} 
 
}

% END OF PREAMBLE

%%%%%%%%%%%%%%%%%%%%%%%%%%%%%%%%%%%%%

\begin{document}

%%%%%%%%%%%%%%%%%%%%%%%%%%%%%%%%%%%%%
  \includepdf{./titlepage/titlepage.pdf}
  \clearpage
  \setcounter{page}{1}
%%%%%%%%%%%%%%%%%%%%%%%%%%%%%%%%%%%%%

\section{Vorbereitungsaufgaben}


\subsection{}
\includegraphics[width=\textwidth]{./1_1/1_1_function.pdf}


%\noindent \textit{Berechnung der Fourierkoeffizienten}
\begin{gather}
	\intertext{Hier gilt}
	x(t) = x(-t), \label{eq:1}\\
	\intertext{weshalb x(t) eine gerade Funktion ist. Damit ist}
	\nonumber
	b_n = \frac{2}{T} \int_T{ x(t) \cdot \sin{( n \omega_0 t)}  \dif t} = 0\\[\smallvert]
	\nonumber
	a_n = \frac{2}{T} \int_T{ x(t) \cdot \cos{( n \omega_0 t)}  \dif t} \\
	\nonumber
	= \frac{2}{T} 
	\left[
		\int_{-T/2}^{-\tau/2}{...}	 +
		\int_{-\tau/2}^{\tau/2}{...}+
		\int_{\tau/2}^{T/2}{...}
	\right]\\
	\intertext{mithilfe von Gl. \ref{eq:1}:}
	\nonumber
	a_n = \frac{2}{T}
	\left[
		2 \int_{0}^{\tau/2}{A \cdot \cos{( n \omega_0 t)}\dif t}+
		2 \int_{\tau/2}^{T/2}{-A \cdot \cos{( n \omega_0 t)}\dif t}
	\right]\\
	\nonumber
	= \frac{4 A}{T}
	\left[
		\int_{0}^{\tau/2}{\cos{( n \omega_0 t)}  \dif t} -
		\int_{\tau/2}^{T/2}{\cos{( n \omega_0 t)}  \dif t}
	\right]\\
	\nonumber
	= \frac{4 A}{T} \cdot \frac{1}{n \omega_0}
	\left[
		\sin{( n \omega_0 t )} \bigg\rvert_{0}^{\tau/2} -
		\sin{( n \omega_0 t )} \bigg\rvert_{\tau/2}^{T/2}
	\right]\\
	\intertext{mit $\omega_0 = \frac{2 \pi}{T}$:}
	\nonumber
	a_n = 
	\frac{4 A \cdot T}{T \cdot n 2 \pi}
	\left[ 
		\sin{(n \frac{2 \pi}{T} \frac{\tau}{2})} - 
		\left(
			\sin{(n \frac{2 \pi}{T} \frac{T}{2})} -
			\sin{(n \frac{2 \pi}{T} \frac{\tau}{2})}
		\right)
	\right]\\
	\nonumber
	= \frac{2 A}{n \pi}
	\left[ 
		\sin{(n \pi \frac{\tau}{T})} - 
		\underbrace{\sin{(n \pi)}}_{=0} +
		\sin{(n \pi \frac{\tau}{T})}
	\right]
	\nonumber
\end{gather}

\eqbox{a_n = \frac{4 A}{n \pi} \cdot \sin{(n \pi \frac{\tau}{T})}}{0.382\textwidth}

\holine{0.618\textwidth}
%\noindent \textit{Berechnung des Gleichanteils}
\begin{gather*}
	a_0 = \frac{1}{T} \int_T{x(t) \dif t}\\
	\intertext{mithilfe von Gl. \ref{eq:1}:}
	a_0 = \frac{1}{T/2} \left[ \int_{0}^{\tau/2}{...} + \int_{\tau/2}^{T/2}{...} \right]\\
	= \frac{2 A}{T} \left[ t \bigg\rvert_{0}^{\tau/2} - t \bigg\rvert_{\tau/2}^{T/2}\right]\\
	= \frac{2 A}{T} \left[ \frac{\tau}{2} - \left( \frac{T}{2} - \frac{\tau}{2} \right)  \right]\\
	= \frac{2 A}{T} \left[ \tau - \frac{T}{2} \right]\\
	= 2 A \left[ \frac{\tau}{T} - \frac{1}{2} \right]
\end{gather*}

\eqbox{\frac{a_0}{2} = A \left[ \frac{\tau}{T} - \frac{1}{2} \right]}{0.382\textwidth}


\holine{\textwidth}

\begin{gather*}
	\intertext{Für das Tastverhältnis $\frac{\tau}{T} = 0.5$  gilt:}
	\frac{a_0}{2} = A \left[ \frac{1}{2} - \frac{1}{2}\right] = 0\\
	a_n = \frac{4 A}{n \pi} \cdot \sin{(n \pi \frac{1}{2})} = \frac{4 A}{n \pi} \cdot \sin{(n \frac{\pi}{2})}\\
	 = \frac{4 A}{n \pi} \cdot (-1)^{n+1}
\end{gather*}

\begin{figure}[H]
	\caption{Betragsspektrum von x(t) für $\frac{\tau}{T} = 0.5$}
\end{figure}

\section{Versuchsaufgaben}

\end{document}
