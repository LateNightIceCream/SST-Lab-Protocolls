\documentclass[a4paper, 12pt]{article}

%%% SST LAB PROTOCOLL PREAMBLE
%%% 2019
%%%%%%%%%%%%%%%%%%%%%%%%%%%%%%%


%%% PACKAGES
%%%%%%%%%%%%%%%%%%%%%%%%%%%

\usepackage{amsmath}
\usepackage{pgfplots}
\usepackage{tikz}
\usepackage{tcolorbox}
\usepackage{graphicx}
\graphicspath{ {./graphics/} }
\usepackage{pdfpages}


%%% DOCUMENT GEOMETRY
%%%%%%%%%%%%%%%%%%%%%%%%%%%

\usepackage{geometry}
\geometry{
 a4paper,
 total={0.6180339887498948\paperwidth,0.6180339887498948\paperheight},
 top = 0.1458980337503154\paperheight,
 bottom = 0.1458980337503154\paperheight
 }
\setlength{\jot}{0.013155617496424828\paperheight}
\linespread{1.1458980337503154}

\setlength{\parskip}{0.013155617496424828\paperheight} % paragraph spacing


%%% COLORS
%%%%%%%%%%%%%%%%%%%%%%%%%%%

\definecolor{red1}{HTML}{f38181}
\definecolor{yellow1}{HTML}{fce38a}
\definecolor{green1}{HTML}{95e1d3}
\definecolor{blue1}{HTML}{66bfbf}
\definecolor{hsblue}{HTML}{0cb5df}
\definecolor{hsblueshade}{HTML}{b6e9f5}


%%% COMMANDS
%%%%%%%%%%%%%%%%%%%%%%%%%%%

\newcommand{\holine}{
  \noindent\rule{\textwidth}{0.618pt}\\[0.021286\paperheight]
}

\newcommand{\minisec}[1]{ \underline{\textit {#1} } \\[0.021286\paperheight]}

\newcommand{\plotfun}[3]{
  \vspace{0.021286\paperheight}
  \begin{center}
    \begin{tikzpicture}
      \begin{axis}[
        axis x line=center,
        axis y line=center,
        ]
        \addplot[draw=red1][domain=#2:#3]{#1};
      \end{axis}
    \end{tikzpicture}
  \end{center}
}

\tcbset{colback=white,colframe=red1!100!black,title=Note!,width=0.618\paperwidth,arc=0pt}
\newcommand{\notebox}[1]{

 \begin{center}
  \begin{tcolorbox}[]
   #1 
  \end{tcolorbox}
 
 \end{center} 
 
}

% END OF PREAMBLE

%%%%%%%%%%%%%%%%%%%%%%%%%%%%%%%%%%%%%

\begin{document}

%%%%%%%%%%%%%%%%%%%%%%%%%%%%%%%%%%%%%
  \includepdf{./titlepage/titlepage.pdf}
  \clearpage
  \setcounter{page}{1}
%%%%%%%%%%%%%%%%%%%%%%%%%%%%%%%%%%%%%



\section{Vorbereitungsaufgaben}

\subsection{}

\begin{figure}[H]
  \includegraphics[width=\textwidth]{1_1/1_1_u1}
  \caption{$u_1(t)$ und $u_2(t)$}
\end{figure}

\begin{gather*}
  u_3(t) = u_1(t) \cdot u_2(t)\\
  = (U_0 + U_1 \cos{(\omega_1 t)}) \cdot U_2 \cos{(\omega_2 t)}\\
  = U_0 U_2 \cos{(\omega_2 t)} + U_1 U_2 [ \cos{(\omega_1 t)} \cdot \cos{(
    \omega_2 t)} ]\\
  \intertext{mithilfe der Gleichung aus "theoretische Grundlagen"\ ergibt
    sich}
  u_3(t)= U_0 \cos{(\omega_2 t)} + \frac{U_1 U_2}{2} \cdot \left[
    \cos{((\omega_1 - \omega_2)t)} + \cos{((\omega_1 + \omega_2) t)} \right]
\end{gather*}

\begin{figure}[H]
  \includegraphics[width=\textwidth]{1_1/1_1_u3}
  \caption{$u_3(t)$; $u_1(t)$ ist als einhüllende Funktion erkennbar}
\end{figure}

\begin{gather*}
	 \intertext{Fouriertransformation:}
U_3(\omega)= U_0 U_2 \cdot \mathcal{F}\{ \cos{(\omega_2 t)} \}+ \frac{U_1 U_2}{2} \cdot \left[
\mathcal{F}\{\cos{((\omega_1 - \omega_2)t)}\} +
\mathcal{F}\{\cos{((\omega_1+\omega_2) t)}\} \right]
\intertext{mit}
%  \mathcal{F}\{ \cos{(2\pi f_0 t)} \} = \frac{1}{2} [ \delta{(f-f_0) + \delta{(f+f_0)}} ]
\mathcal{F}\{ \cos{(\omega_0 t)} \} = \frac{1}{2} [ \delta{(\omega-\omega_0) + \delta{(\omega+\omega_0)}} ]
\intertext{ergibt sich}
	\begin{split}
	U_3(\omega) = \frac{U_0 U_2}{2}\cdot [ \delta(\omega-\omega_2) + \delta(\omega + \omega_2) ] \\ + \frac{U_1 U_2}{4} \cdot [ \delta(\omega-(\omega_1-\omega_2)) \\+ \delta(\omega + \omega_1-\omega_2) \\+ \delta(\omega-(\omega_1+\omega_2)) \\+ \delta(\omega + \omega_1 + \omega_2)]
	\end{split}
\intertext{mit den vorgegebenen Werten:}
	\begin{split}
U_3(f) = 0.3 \cdot [ \delta(f-4\si{\kilo\hertz} + \delta(f + 4 \si{\kilo\hertz}) ] \\ + \frac{1}{4} \cdot [ \delta(f+3.5 \si{\kilo\hertz}) \\+ \delta(f - 3.5 \si{\kilo\hertz}) \\+ \delta(\omega-4.5 \si{\kilo\hertz}) \\+ \delta(\omega + 4.5 \si{\kilo\hertz})]
\end{split}
\end{gather*}

\begin{figure}[H]
  \includegraphics[width=\textwidth]{1_1/1_1_spektrum}
  \caption{Amplitudendichtespektrum von $u_3(t)$}
\end{figure}

%%%%%%%%%%%%%%%%%%%%%%%%%
\subsection{}

\begin{figure}[H]
  \includegraphics[width=\textwidth]{1_2/u1_carrier}
  \caption{$u_1(t)$ und $u_2(t)$}
\end{figure}

\begin{figure}[H]
  \includegraphics[width=\textwidth]{1_2/u3}
  \caption{$u_3(t)$ aus der Multiplikation von $u_1(t)$ mit dem Träger $u_2(t)$}
\end{figure}

\noindent Fouriertransformation:

Die multiplikative Verknüfung der Signale $u_1(t)$ und $u_2(t)$ im Zeitbereich
lässt sich eine Faltung im Frequenzbereich überführen. Da dies analytisch nicht
trivial ist, wird hier auf eine grafische Lösung zurückgegriffen.\\ Dazu die Vorüberlegung:

\begin{gather*}
  x_1(t) = \delta{(t-T)} \,\ \laplace \,\ {X_1(f) = e^{-j \cdot 2\pi \cdot f \cdot T}}\\ 
\end{gather*}

\subsection{Praktische Anwendungsmöglichkeiten}

Die Multiplikation von Zeitsignalen findet Anwendung in der Nachrichtenübertragung. Dort kann das zu übertragene Signal mit einem \emph{Trägersignal} multipliziert werden, um zum einen das Signal in einen für die Antennentechnik brauchbaren\footnote{Änderung der Übertragungseigenschaften des Signals mit der Frequenz} Frequenzbereich zu bringen und zum anderen das Signal am Empfänger von anderen Signalen in der Frequenz unterscheiden zu können (vgl. 1.1).

Außerdem nutzen Oszilloskope die Multiplikation mit einem Rechtecksignal um ein Fenster des aufgenommenen Signals zu erhalten, welches dann fourieranalysiert werden kann.

\section{Versuchsaufgaben}

\subsection{}

\begin{figure}[H]
  \includegraphics[width=\textwidth]{2_1/scope_5_blue}
  \caption{FFT von $u_3(t)$ analog Aufgabe 1}
\end{figure}

Wie in Abbildung X zu sehen, stimmen die ermittelten Frequenzen mit denen der
Vorbereitungsaufgabe überein.
Durch die Limitierungen des Oszilloskops (z.B.
Fensterung) und der Soundkarte sowie Störeinflüsse sieht man keine exakt
diskreten Amplitudenlinien, die Maxima sind jedoch klar bei den berechneten
Frequenzen erkennbar.

\begin{figure}[H]
  \includegraphics[width=\textwidth]{2_1/scope_6_blue}
  \caption{FFT von $u_3(t)$ ohne Gleichanteil der Nachricht}
\end{figure}

Nach Entfernen des Gleichanteils des Nachrichtensignals verschwindet die Frequenz des
Träges aus dem Betragsspektrum des übertragenen Signals (Abbildung X). Der Gleichanteil der Nachricht lässt also den Träger erscheinen. 


Wird nun das Trägersignal mit einem Gleichanteil versehen, wird die Originalfrequenz des Nachrichtensignals mitübertragen.


Erhöht man jetzt die Frequenz des Nachrichtensignals, rücken diese und die untere Frequenz des multiplizierten Signals immer näher zusammen. Die multiplizierten Frequenzen rücken dabei weiter auseinander. Bei einer Frequenz von $2 \,\ \si{\kilo\hertz}$ überlappen sich die Originalfrequenz und die multiplizierte Frequenz und sind nicht mehr unterscheidbar.

Bei Frequenzen, die sehr nahe an der Überschneidungsfrequenz sind (z.B. $1990 \,\ \si{\hertz}\}$) entstehen Schwebungen wie in Abb. x zu sehen.

\subsection{}
Wie in Abbildung x zu sehen, stimmen die ermittelten Frequenzen mit denen der Vorbereitungsaufgabe überein. Durch die limitierte Fensterung und Abtastung des Signals durch das Oszilloskop sieht man keine diskreten Amplitudenlinien, sondern 222

Bei Frequenzerhöhung rücken, wie auch in Versuchsaufgabe 2.1, die untere multiplizierte Frequenz und die Frequenz des Nachrichtensignals näher aneinander, bis sie sich bei $2 \,\ \si{\kilo\hertz}$ treffen.

Die Zäune können als veschiedene Übertragungsbreiten aufgefasst werden


Wie auch in der Versuchsaufgabe 2.1 lässt ein hinzugefügter Gleichanteil im Träger- bzw. Nachrichtensignal die Originalfrequenz des jeweils anderen Signals im Spektrum erscheinen.


\end{document}
