\documentclass[a4paper, 12pt]{article}

%%% SST LAB PROTOCOLL PREAMBLE
%%% 2019
%%%%%%%%%%%%%%%%%%%%%%%%%%%%%%%


%%% PACKAGES
%%%%%%%%%%%%%%%%%%%%%%%%%%%

\usepackage{amsmath}
\usepackage{pgfplots}
\usepackage{tikz}
\usepackage{tcolorbox}
\usepackage{graphicx}
\graphicspath{ {./graphics/} }
\usepackage{pdfpages}


%%% DOCUMENT GEOMETRY
%%%%%%%%%%%%%%%%%%%%%%%%%%%

\usepackage{geometry}
\geometry{
 a4paper,
 total={0.6180339887498948\paperwidth,0.6180339887498948\paperheight},
 top = 0.1458980337503154\paperheight,
 bottom = 0.1458980337503154\paperheight
 }
\setlength{\jot}{0.013155617496424828\paperheight}
\linespread{1.1458980337503154}

\setlength{\parskip}{0.013155617496424828\paperheight} % paragraph spacing


%%% COLORS
%%%%%%%%%%%%%%%%%%%%%%%%%%%

\definecolor{red1}{HTML}{f38181}
\definecolor{yellow1}{HTML}{fce38a}
\definecolor{green1}{HTML}{95e1d3}
\definecolor{blue1}{HTML}{66bfbf}
\definecolor{hsblue}{HTML}{0cb5df}
\definecolor{hsblueshade}{HTML}{b6e9f5}


%%% COMMANDS
%%%%%%%%%%%%%%%%%%%%%%%%%%%

\newcommand{\holine}{
  \noindent\rule{\textwidth}{0.618pt}\\[0.021286\paperheight]
}

\newcommand{\minisec}[1]{ \underline{\textit {#1} } \\[0.021286\paperheight]}

\newcommand{\plotfun}[3]{
  \vspace{0.021286\paperheight}
  \begin{center}
    \begin{tikzpicture}
      \begin{axis}[
        axis x line=center,
        axis y line=center,
        ]
        \addplot[draw=red1][domain=#2:#3]{#1};
      \end{axis}
    \end{tikzpicture}
  \end{center}
}

\tcbset{colback=white,colframe=red1!100!black,title=Note!,width=0.618\paperwidth,arc=0pt}
\newcommand{\notebox}[1]{

 \begin{center}
  \begin{tcolorbox}[]
   #1 
  \end{tcolorbox}
 
 \end{center} 
 
}

% END OF PREAMBLE

%%%%%%%%%%%%%%%%%%%%%%%%%%%%%%%%%%%%%

\begin{document}

%%%%%%%%%%%%%%%%%%%%%%%%%%%%%%%%%%%%%
  \includepdf{./titlepage/titlepage.pdf}
  \clearpage
  \setcounter{page}{1}
%%%%%%%%%%%%%%%%%%%%%%%%%%%%%%%%%%%%%



\section{Vorbereitungsaufgaben}

\subsection{}

Hier noch ne grafik der funktionen

\begin{gather*}
  u_3(t) = u_1(t) \cdot u_2(t)\\
  = (U_0 + U_1 \cos{(\omega_1 t)}) \cdot U_2 \cos{(\omega_2 t)}\\
  = U_0  \cos{(\omega_2 t)} + U_1 U_2 [ \cos{(\omega_1 t)} \cdot \cos{(
    \omega_2 t)} ]\\
  \intertext{mithilfe der Gleichung aus "theoretische Grundlagen"\ ergibt
    sich}
  u_3(t)= U_0 \cos{(\omega_2 t)} + \frac{U_1 U_2}{2} \cdot \left[
    \cos{((\omega_1 - \omega_2)t)} + \cos{((\omega_1 + \omega_2) t)} \right]\\
  \intertext{Fouriertransformation:}
  U_3(\omega)= U_0 \cdot \mathcal{F}\{ \cos{(\omega_2 t)} \}+ \frac{U_1 U_2}{2} \cdot \left[
    \mathcal{F}\{\cos{((\omega_1 - \omega_2)t)}\} +
    \mathcal{F}\{\cos{((\omega_1+\omega_2) t)}\} \right]
  \intertext{mit}
%  \mathcal{F}\{ \cos{(2\pi f_0 t)} \} = \frac{1}{2} [ \delta{(f-f_0) + \delta{(f+f_0)}} ]
  \mathcal{F}\{ \cos{(\omega_0 t)} \} = \frac{1}{2} [ \delta{(\omega-\omega_0) + \delta{(\omega+\omega_0)}} ]
  \intertext{ergibt sich}
\end{gather*}


\subsection{}

\subsection{}

\section{Versuchsaufgaben}

\subsection{}

\subsection{}

\subsection{}


\end{document}
